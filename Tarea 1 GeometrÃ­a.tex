% Title of Document:
% Version:
% Start Date:      
% Revision Date:      
%%%%%%%%%%%%%%%%%%%%%%%%
\documentclass[12pt,answers]{exam}

% do not print answers  
\noprintanswers

\usepackage{amsmath,amsfonts,amssymb}

\usepackage[spanish]{babel}

\usepackage[utf8]{inputenc}

% Redifine Solution -> Solución
\renewcommand{\solutiontitle}{\noindent\textbf{Solución:}\enspace}

% Redifine Point Points -> Punto Puntos 
\pointpoints{punto}{puntos}

%Pdf Graphicx
\usepackage[pdftex]{graphicx,color} 
\usepackage{graphicx}



\textwidth=18truecm
\textheight=25truecm
\topmargin=-50pt
\oddsidemargin=-20pt
\evensidemargin=-20pt


\begin{document}
\begin{center}
\bf
GEOMETRÍA DIFERENCIAL 201620
\\
Tarea 1.  semana 1 (3/08 -- 5/03).
\end{center}
\bigskip

\underline{\emph{La fecha para entregar la solución es el miércoles  10/08/2016}} 
\bigskip

\begin{questions}

\question[1]
Sea $\mathbb{S}^2 = \left\{ (x, y, z) \in \mathbb{R}^3 \mid (x^1)^2 + (x^2)^2 + (x^3)^2 = 1 \right\}$ la esfera unitaria en $\mathbb{R}^3$. 
Pongamos,
\begin{equation*}
\begin{array}{l}
U_3^+ = \left\{ (x^1, x^2, x^3) \in \mathbb{S}^3 \mid x^3 > 0  \right\}, \quad \varphi_3^+(x^1, x^2, x^3) = (x^1, x^2)
\\
U_3^- = \left\{ (x^1, x^2, x^3) \in \mathbb{S}^3 \mid x^3 < 0  \right\}, \quad \varphi_3^-(x^1, x^2, x^3) = (x^1, x^2)
\\
U_2^+ = \left\{ (x^1, x^2, x^3) \in \mathbb{S}^3 \mid x^2 > 0  \right\}, \quad \varphi_2^+(x^1, x^2, x^3) = (x^3, x^1)
\\
U_2^- = \left\{ (x^1, x^2, x^3) \in \mathbb{S}^3 \mid x^2 < 0  \right\}, \quad \varphi_2^-(x^1, x^2, x^3) = (x^3, x^1)
\\
U_1^+ = \left\{ (x^1, x^2, x^3) \in \mathbb{S}^3 \mid x^1 > 0  \right\}, \quad \varphi_1^+(x^1, x^2, x^3) = (x^2, x^3)
\\ 
U_1^- = \left\{ (x^1, x^2, x^3) \in \mathbb{S}^3 \mid x^1 < 0  \right\}, \quad \varphi_1^-(x^1, x^2, x^3) = (x^2, x^3)
\\
\end{array}
\end{equation*} 
Probar que $\mathcal{A} = \left\{ (U_i^{\pm}, \varphi_i^{\pm}) \right\}$ es un atlas sobre $\mathbb{S}^2$. 



\question[1]
Construir un atlas sobre la esfera $n$-dimensional  $\mathbb{S}^n = \left\{ x \in \mathbb{R}^{n + 1} \mid \| x \| = 1 \right\}$ que consiste de dos cartas. Ayuda: usar la proyección estereográfica.  


\question[1]
Probar que la topología determinada por la estructura de variedad sobre $\mathbb{S}^2$ (ver los ejercicios 1 y 2) coincide con la topología inducida por la inclusión $\mathbb{S}^2 \subset \mathbb{R}^3$.   



\question[1]
Sean $(M_1, \mathcal{A}_1)$ y  $(M_2, \mathcal{A}_2)$ dos variedades.
Probar que

a) Para dos cartas  $c_1 = (U_1, \varphi_1) \in \mathcal{A}_1$ y $c_2 = (U_2, \varphi_2) \in \mathcal{A}_2$, la pareja $c_1 \times c_2 = (U_1 \times U_2, \varphi_1 \times \varphi_2)$ es una carta sobre $M_1 \times M_2$.

b) El conjunto de cartas $\mathcal{A}_1 \times \mathcal{A}_2 = \left\{ c_1 \times c_2 \mid c_1 \in \mathcal{A}_1, c_2 \in \mathcal{A}_2 \right\}$ es un atlas sobre $M_1 \times M_2$. 

La variedad $M_1 \times M_2 = (M_1 \times M_2, \mathcal{A}_1 \times \mathcal{A}_2)$ se llama el \emph{producto de las variedades $M_1$ y $M_2$}.

\question[1]
Sean $M_1$ y $M_2$ variedades.
Probar que 

a) La proyección $\pi_1 : M_1 \times M_2 \to M_1$, $\pi_1(p_1, p_2) = p_1$, es una aplicación suave.

b) Una aplicación $f : N \to M_1 \times M_2$, donde $N$ es una variedad, es suave si y sólo si $\pi_1 \circ f : N \to M_1$ y $\pi_2 \circ f : N \to M_2$ son aplicaciones suaves.  


\end{questions}


\end{document}
